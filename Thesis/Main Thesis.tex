\documentclass{article}
\usepackage{graphicx} % Required for inserting images
\usepackage{hyperref}

\title{The genetics of Mitochondrial Dysfunction in Sporadic Parkinson's disease}
\author{Thomas Brightwell; Supervisors: Toby Andrew and Rahel Feleke}

\date{August 2024}

\begin{document}

\maketitle
This project
report is submitted in partial fulfilment of the requirements for award of MSc in Human Molecular
Genetics, Imperial College London. I hereby confirm that it is an original work, representing my
own academic effort and that all sources have been fully acknowledged.
\newpage
\begin{abstract}
\end{abstract}
\renewcommand{\abstractname}{Acknowledgements}
\begin{abstract}
\end{abstract}
\newpage
\tableofcontents
\section{Abbreviations}
PD Parkinson's Disease
\\SN Substantia Nigra
\\LBs Lewy-Bodies
\\fPD familial Parkinson's Disease
\\sPD sporadic Parkinson's Disease
\\MSA Multiple System Atrophy
\\GWAS Genome Wide Association Study
\\GWASs Genome Wide Association Studies
\\GTEx Project Genotype Tissue Expression Project
\\SNP Single Nucleotide Polymorphism
\\PRS Polygenic Risk Score
\\LD Linkage Disequilibrium
\\LDU Linkage Disequilibrium Units
\\ETC Electron Transport Chain
\\QTL Quantitative Trait Locus
\\QTLs Quantitative Trait Loci
\\eQTL expression Quantitative Trait Locus
\\sQTL splicing Quantative Trait Locus
\\GSEA Gene Set Enrichment Analysis
\\PPMI Parkinson's Progression Markers Initiative
\\TAD Topologically Associated Domain
\\ROS Reactive Oxygen Species
\\DEGs Differentially Expressed Genes
\\GO Gene Ontology
\\NEMG Nuclear Encoded Mitochondrial Gene
\\GEO Gene Expression Omnibus
\\PPMI Parkinson's Progression Markers Initiative
\\OMM Outer Mitochondrial Membrane
\\IMM Inner Mitochondrial Membrane
\\iPSC induced Pluripotent Stem Cell
\\BRAINEAC Brain Expression Almanac
\\PMI Post Mortem Index
\\RIN RNA Integrity Number
\section{Introduction}
\subsection{Parkinson's Disease}
\subsubsection{Background}  
Parkinson's Disease (PD), is a neurodegenerative disorder named after James Parkinson, who first formally described the disease in his 1817 "An Essay on the Shaking Palsy"\cite{Parkinson2002AnPalsy}. The disease is primarily phenotypically characterized by tremors, bradykinesia. rigidity, and postural instability, although other non-motor symptoms are associated with the disease. These motor symptoms are caused by the fundamental pathophysiology of PD: the death of dopaminergic neurons in the Substantia Nigra (SN), a part of the basal ganglia. PD is the second most common neurodegenerative disorder, behind Alzheimer's Disease, and the 2021 Global Health Data exchange\cite{Ferrari2024Global2021} has estimated global statistics for PD. Yearly incidence is estimated to be 1.3 million cases. Global incidence is estimated at 0.15\%, but the age related nature of PD can clearly be seen when this is split into the under and over 70 categories, with a prevalence of 0.06\% and 1.45\% respectively. 
This has profound implications for countries with aging populations like the UK, where the incidence has increased by 41\% in the last 30 years \cite{Ferrari2024Global2021}. There is great impetus to understand the underlying molecular aetiology of PD to improve treatments, and resolve the increasing global burden.
At the core of PD and many other conditions group together as "synuceinopathies" is the protein $\alpha$-synuclein and its aggregates. Lewy-Bodies (LBs) are intracellular inclusions formed from $\alpha$-synuclein\cite{Spillantini1997-SynucleinBodies}, and are an important hallmark of PD. However, despite their widespread nature across PD subtypes and other diseases, the mechanism by which LBs causes pathological symptoms (if it does at all) is not fully clear\cite{Riederer2023LewyDisease}. It is however clear, that $\alpha$-synuclein itself is a crucial factor for PD.
\subsubsection{Clinical diagnosis of sPD}
\subsubsection{PD sub-types and terminology}
PD is often distinguished clinically by the age of onset, but the definition of early or late onset is not a consensus. Usually, those under 50 years are considered to have early-onset Parkinson's, with those who develop later symptoms considered to have late-onset PD, but some use a threshold of 40 years instead\cite{Ferguson2016Early-onsetStudy}.
Parkinson's Disease has 3 genetic classifications: Familial Parkinson's Disease (fPD), Sporadic Parkinson's Disease (sPD), and GBA1-type Parkinson's Disease (GPD). fPD is a monogenic disorder, with certain key genes being responsible for nearly all cases of fDP. Depending on the gene responsible for the specific patient's fPD, it can inherited in either a autosomal dominant or autosomal recessive fashion\cite{Day2021ThePractice}. GPD is separated from standard fPD because it does not follow a standard monogenic inheritance pattern and has very variable penetrance.
sPD (sometimes also called idiopathic PD) is a complex trait, with considerable environmental influence and a heritable component\cite{Nalls2019IdentificationStudies}. Many different environmental factors have been found to influence sPD risk\cite{Costa2023ParkinsonsDisorder}, from pesticide exposure to caffeine intake, however age is the most important. sPD accounts for the majority of PD cases - up to 90\% of PD patients do not have a family history of the disease\cite{Inamdar2007ParkinsonsBeyond}. This means that most cases of PD do not have a clear cause. Additionally, there is significant overlap between patients who have a "Parkinsonism", for example those who have Lewy-Body dementia first, and develop PD-like symptoms afterwards\cite{Jellinger2018DementiaControversies}, and those who have PD first and develop other symptoms later.
There are many different Parkinsonisms - and these are still a subset of synucleiopathies that include diseases such as Multiple System Atrophy (MSA)\cite{Hayes2019ParkinsonsParkinsonism}. These diseases have a shared aetiology of the $\alpha$-synuclein protein misfolding.
Whilst fPD can be mitigated in the population by genetic counselling and careful planning, sPD is intrinsically harder to predict and prevent. This study is specifically investigating the genetic component of sPD, in an effort to unpick the aetiology of the risk variants described in Nalls. et al.\cite{Nalls2019IdentificationStudies}. Understanding the molecular nature of common disease has the potential to highlight important variants and genes that can be targeted for further functional validation, and if promising, therapeutic treatment.
\subsubsection{The genetics of Parkinson’s Disease}
PD is a disease with great genetic heterogeneity, along with the previously discussed clinical heterogeneity. 
\paragraph{Genetics of fPD}Up to 17 genes have been discovered as causes of fPD\cite{Day2021ThePractice}, with most following an AD pattern of inheritance, and a minority following a AR pattern. The first discovered fPD gene was SNCA, which codes for $\alpha$-synuclein, despite the variants actual rarity. In contrast, the most common fPD mutations are in the LRRK2 gene\cite{Klein2012GeneticsDisease}. Each of the genes identified as causing fPD offers insight into the mechanisms underlying PD, and many of these genes are associated with the mitochondria (discussed in section 2.2).
\begin{table}[h]
    \begin{tabular}{ |c|c|c|c|c|c|c|c|c| }
        \hline
        Gene Symbol & Gene name & Chromosome & Pathway(s) & Notes \\
        SNCA & $\alpha-$synuclein \\
        LRRK2 \\
        GBA1 \\
        PINK1 \\
        PRKN & Parkin \\
        PARK7
        
         \\
         \hline
    \end{tabular}
    \caption{Table 1: Key familial Parkinson's disease }
    \label{tab:my_label}
\end{table}
\paragraph{sPD}The genetics of sPD are much less clear than fPD. Many individual studies have tried to establish what can explain the heritable factor of sPD, but the most successful is a 2019 meta-analysis of PD-GWASs by Nalls. et al. \cite{Nalls2019IdentificationStudies}. In their study, they identified 90 lead SNPs most significantly associated with PD. This meta analysis combined 16 databases of case-control data for PD, with a total of 37.7K cases, and 1.4M controls. They employed a conditional and joint analysis strategy to determine what variants best explained the heritability of PD. They estimated a PRS that explained a median 26\% of the heritability for PD. However, the authors do note that this is highly dependent on the prevalence of PD in the population, and on the inclusion of exclusion of rare variants within the calculation.
\subsection{The role of mitochondria in PD}
\subsubsection{Mitochondrial Background}
The mitochondria is a double-membrane bound organelle , common to almost all eukaryotes (aside from a few unicellular parasites that are theorized to have lost them\cite{Karnkowska2016AOrganelle}). It is now widely accepted that the origin of the mitochondria was an endosymbiotic event\cite{Martin2015EndosymbioticOrigin.}, where the ancestor to eukaryotic cells internalized a bacterial cell and began to work in concert with it to the benefit of both cells. As it has an extracellular origin, the mitochondria has a key feature that other non-nuclear organelles lack: its own genome. The mitochondrial genome has been known to be very small for a long time\cite{Taanman1999TheReplication}, encoding only 37 genes. However, there are many more genes not encoded by the mitochondrial genome that still localise to, and perform their cellular role at, the mitochondria. The latest version of MitoCarta\cite{Rath2021MitoCarta3.0:Annotations} has 1136 human protein-coding genes that have strong evidence for localization to the mitochondria, highlighting how the overwhelming majority of mitochondrial genes have migrated to the nucleus over the course of evolution.
Mitochondria are involved in many cellular processes, but some of the key pathways for disease are ATP production by oxidative phosphorylation, apoptosis, and cell signalling\cite{Rossmann2021MitochondrialDisease}. Additionally, the production of Reactive Oxygen Species (ROS) as a byproduct of oxidative phosphorylation is an important factor for both aging and disease\cite{Brieger2012ReactiveDisease}. Diseases linked with mitochondria can therefore have a wide range of phenotypes - from metabolic disorders\cite{Bhatti2017MitochondrialStrategies} to autoimmune diseases\cite{Xu2020EmergingDiseases}, and most importantly for this study, neurodegenerative disorders\cite{MonzioCompagnoni2020TheDisease}.
\subsubsection{$\alpha$-synuclein}
As described in 2.1.1, $\alpha$-synuclein is fundamental to PD and other synucleinopathies, but its role in both healthy function and disease state in the brain is not fully understood, although great progess is being made. There are multiple forms that $\alpha$-synuclein can take, from single monomers, to oligomers, to full amyloid fibrils\cite{Mehra2019-SynucleinPathogenesis}. Soluble monomers of $\alpha$-synuclein binds to lipid membranes, which includes both vesicles and the mitochondrial membrane\cite{Burre2018Cell-Synuclein}. It has been shown that the "seeding" events that convert functional $\alpha$-synuclein into the oligomeric form that acts as a toxin\cite{Choi2022PathologicalToxicity}.
Another way that $\alpha$-synuclein is thought to act linked is through inflammation, as $\alpha$-synuclein can trigger the NLRP3 inflammasome complex\cite{Forloni2023AlphaInflammation}. $\alpha$-synuclein has been proposed to play a role in many different PD-linked pathways, some of are mitochondrial, will be discussed in 2.2.3
\paragraph{Inflammation}
\paragraph{Is this relevant? Having a hard time placing where this should go exactly.}
Chronic inflammation is observed in PD brains, as cell death leaves debris in the surrounding tissue, attracting microglia to clear the debris away\cite{Pajares2020InflammationImplications}. However, microglia can also promote cell death via apoptosis, and are known to be key to synaptic pruning. There is evidence for a feedback loop between these events, where cell death creates debris, attracting microglia and causing inflammation, leading to greater cell death and repeating the cycle. Additionally, $\alpha$-synuclein aggregates are thought to also trigger an inflammatory response, and it has been proposed that this explains the non-motor symptoms that can occur decades earlier in patients in areas like the gut\cite{Forloni2023AlphaInflammation}. The cause and effect however is not clearly established, and inflammation may be more of a symptom than a cause of PD, but there is strong evidence for inflammation modulate the spread of the disease across neurons.
\subsubsection{Mitochondria in PD}
The first evidence for mitochondrial involvement in PD emerged in 1983, when a complex I inhibitor produced as a byproduct of heroin production was shown to cause PD-like symptoms\cite{Langston1983ChronicSynthesis}, and the mitochondria is now considered one of the most important factors in both fPD and sPD\cite{Henrich2023MitochondrialPotential}, and many of the genes responsible for fPD are mitochondrial. Whilst there are many (and not mutually exclusive) different theories for the specifics of how mitochondria are involved in the aetiology of PD, some of the of the most convincing are outlined here: Mitophagy, Mitochondrial Dynamics, and Oxidative stress.
\paragraph{Mitophagy}Two genes, PINK1 and PRKN (which codes for the Parkin protein), are common causes of fPD when mutated\cite{Malpartida2021MitochondrialTherapy}. PINK1 is a kinase localized to the mitochondrial membranes\cite{Narendra2010PINK1Parkin}, and Parkin is a cytosolic ubiquitin ligase\cite{Narendra2008ParkinAutophagy}. Both of these genes interact together to recognise and flag impaired mitochondria for degradation. Under normal conditions, PINK1 is cleaved by factors in the mitochondrial matrix and Inner Mitochondrial Membrane (IMM), allowing it to degraded by the proteosome\cite{Pickles2018MitophagyMaintenance}. However, under stress conditions, PINK1 becomes stabilized on the Outer Mitochondrial Membrane (OMM), and can perform its function of phosphorylating Parkin. Once phosphorylated, Parkin can then be recruited to the OMM, where it acts as a signal for mitophagy. As mutations in either of these genes can cause fPD it implies that mitochondrial degradation and mitophagy plays an important role in PD. Additionally, other PD genes like GBA, SNCA, and LRRK2 all interact with this pathway in some form to impair affect when mutated\cite{Malpartida2021MitochondrialTherapy}. The mechanistic details are less clear, but impaired mitochondrial homeostasis may lead to an increase of stress factors on the cell, leading to apoptosis\cite{Eldeeb2022MitochondrialDisease}.
\paragraph{Mitochondrial dynamics}Mitochondrial dynamics are important for maintaining the delicate balance of a neuronal cell\cite{Chen2009MitochondrialDiseases}. Due to the unique morphology and energetic requirements of neurons, a cell keeping the right number of mitochondria in the right locations is a complex and tightly regulated process. The two previously mentioned genes PINK1 and PRKN have a role in determining mitochondrial fission in addition to the earlier described role in mitophagy, and mitochondrial dynamics and mitophagy are inherently linked\cite{Archer2013MitochondrialDiseases}. Disrupted fission processes increases oxidative stress on the cell and can lead to apoptosis. LRRK2, another fPD gene promotes mitochondrial fission by phosphorylation of DRP1, and the most common single cause of fPD is a specific G2019S mutation that causes over-activity and therefore aberrant mitochondrial fission\cite{Su2013InhibitionMutation}. $\alpha$-synuclein is also thought to play a role in many mitochondrial dynamics processes, including inhibition of fusion and mitochondrial transport\cite{Valdinocci2019IntracellularDisease}. Overall, there is clear evidence that mitochondrial dynamics are central to PD, although the mechanisms for this action are not fully understood yet.
\paragraph{Oxidative Stress}
ROS are well known to be associated with many poor outcomes for cells, and excess ROS can cause major damage to both organelles and macro-molecules, which leads to importance in many different diseases, including neurodegenerative diseases\cite{Brieger2012ReactiveDisease}. Normally, cells that accumulate too much damage undergo apoptosis, and this may be the cause for the characteristic death of dopaminergic neurons in the SN\cite{Subramaniam2013MitochondrialDisease}. Many of the toxins that can trigger onset of PD or PD-like symptoms are inhibitors of some factor in the Electron Transport Chain (ETC), most commonly complex 1\cite{Subramaniam2013MitochondrialDisease}.
Complex I is one of two main locations for the production of Reactive Oxygen Species (ROS) in the mitochondria (and therefore the cell): complexes I\&III\cite{Murphy2009HowSpecies}. Complex I is is a common thread between many different genes that cause fPD. SNCA, PINK1, PARK2,7\&8 are all fPD genes shown to cause complex I impairment\cite{Subramaniam2013MitochondrialDisease}. The complex I inhibitor Rotenone has been used to create PD models in rodents since the 2000s\cite{Betarbet2000ChronicDisease}, and there is evidence that specific disruption of complex I in dopaminergic neurons reproduces PD symptoms\cite{Gonzalez-Rodriguez2021DisruptionParkinsonism}. It is thought that disruption of complex I leads to an increased production of ROS in the cell, which increases the stress on the cell, leading to cell death\cite{Subramaniam2013MitochondrialDisease}.
The toxic oligomeric form of $\alpha$-synuclein also leads to the generation of ROS, which can then go on to damage other critical cell compartments \cite{Choi2022PathologicalToxicity}, and there is evidence for a feedback loop between $\alpha$-synuclein, ROS, iron and neuromelanin, which propogates the spread of PD in the brain\cite{JansenvanRensburg2021ToxicTurmeric}.
The increase in ROS also damages the mitochondrial genome, which may lead to a build-up of deleterious mutations. It has been shown that PD patients have higher frequencies of mtDNA deletions in the SN than other regions of the brain\cite{Bender2006HighDisease}.
Together, these do make a strong case for dysfunction in the ETC being heavily involved in the aetiology of PD.
\subsection{Genetic analysis techniques}
\subsubsection{Linkage disequilibrium and association mapping}
\label{subsubsec:linkage}
Linkage is the phenomenon of genetic loci that are physically close to each other on the same chromosome tending to be inherited together, disobeying the law of independent assortment. This occurs because recombination has a very low chance of occurring between two loci that are next to each other,  so linked alleles are typically inherited together on a single haplotype. Linkage analysis relies on this to track disease loci without directly knowing its location, as there are many commercially available sequencing chips that contain markers alleles spread across the genome. At least one marker should be in close proximity to the actual disease loci, so they will be linked. Therefore, we can analyse the patterns of inheritance of the disease phenotype and marker alleles to determine which markers are linked with the disease loci, and therefore the approximate location of the disease loci.
The first evidence for a genetic role of specific genes in PD disease was discovered through linkage analysis on affected families, which revealed several genes implicated in fPD\cite{Abbas1999AEurope,Maher2002SegregationGene,West2004GeneticsDisease}.
\subsubsection{QTL}
\label{subsubsec:QTL}
A Quantitative Trait Locus (QTL) is a regions of the genome that is associated with, and therefore potentially explain, a difference in a quantitative trait between individuals. A QTL can be linked to any quantitative trait, but the trait= considered in this study is gene expression, and regions affecting them are referred to as an expression Quantitative Trait Locus (eQTL).
This analysis works by combining genotype information across a sample with the quantitative trait information for that sample (in this case, mRNA transcript levels), and testing each specific locus for association with a change in the quantitative trait. 
\subsubsection{Co-location}
The use of QTLs in this study is to implicate causal genes for the lead SNPs found in the Nalls. et al.\cite{Nalls2019IdentificationStudies} meta analysis. This is achieved by co-locating the QTL with the disease loci - which means that they are both in the same genetic location. If the same region has both an association with a disease, and with a change in gene expression, then the hypothesis is that said change in gene expression is causing the association with disease. We will refer to QTLs in the same location as the disease loci as "disease-QTLs". If you then consider which genes have their expression altered by these disease-QTLs, they are good candidates for being causal genes in the disease. 
There are multiple ways to conduct co-localisation, but this study uses genetic maps from\cite{Maniatis2004PositionalDisequilibrium.}, and follows a similar method to that described by Maude et al.\cite{Maude2021NewDiabetes.}, to search for QTLs within the physical coordinates of the LD block around each of the lead SNP.
\subsubsection{Pathway Analyses}
The biological interpretation of multiple individual genes identified in a study can be very difficult, especially when conducting a genome-wide study with potentially hundreds of significant genes. One approach to solving this problem is to reduce the complexity of your results by employing strategies that cluster genes together. Two main approaches for this clustering are network and pathway analysis. Network analyses cluster genes based on their interactions, creating networks of genes encoding proteins that all have physical interactions with each other\cite{Maayan2011IntroductionBiology}. Pathway analyses instead cluster genes based on shared functions and roles in the cell\cite{Garcia-Campos2015PathwayArt} - for example, grouping all genes that code for a part of the TCA cycle into specific TCA cycle group.
There are many different database that can be used for clustering genes in a pathway analysis, such as Gene Ontology (GO)\cite{Ashburner2000GeneBiology}, or the Kyto Encyclopedia of Genes and Genomes (KEGG)\cite{Kanehisa2016KEGGAnnotation}. There are also many different tools to analize your set of genes in comparison to the chosen database, and in this study we will be using Gene Set Enrichment Analyis (GSEA)\cite{Subramanian2005GeneProfiles}. See section 3.6 for more details.
\subsection{Study Aims}
\subsubsection{Hypothesis}
Mitochondrial (dys)function is a key player in the aetiology of neurological diseases\cite{Bartman2024MitochondrialDiseases}, and particularly neurodegenerative diseases such as Alzheimer's and PD\cite{MonzioCompagnoni2020TheDisease}. As discussed in part 2.2, there is clear evidence showing potentially multiple different pathways modulating PD in patients. However, what is less clear is the role genetics may have in shaping mitochondrial dysfunction in PD. We therefore hypothesise that "Mitochondrial dysfunction plays a major role in the onset and 
pathophysiology of sporadic Parkinson’s Disease via multiple identifiable molecular pathways". 
\subsubsection{Aims}
There are four aims of this study:
First, to define the regions of linkage disequilibrium around each of the lead SNPs identified by Nalls et al.\cite{Nalls2019IdentificationStudies} using genetic maps. The physical coordinates of this region will be defined as the 'LD block' that contains the 
Secondly, to harvest all the SNPs in high LD with the lead SNPs within that genetic region, and characterise any coding or splicing variants that may be contributing to the GWAS signal.
Thirdly, to conduct disease-eQTL co-location analyses to identify cis-genes that may be affected by the identified SNPs in both healthy and PD affected individuals.
Finally, to use expression data to conduct pathway analysis and GSEA\cite{Subramanian2005GeneProfiles} to test for enrichment of mitochondrial pathways in general PD patients, and specifically the enrichment of NEMGs in the disease-affected cis-genes we identified in the third aim.
\section{Materials and Methods}
\subsection{Linkage disequilibrium - the key to gene mapping}
As described in section \hyperref[subsubsec:linkage]{2.3.1}, linkage is the property of two alleles in close proximity tending to be inherited together. Linkage Disequilibrium (LD) is a related, but distinct, phenomenon of two or more genetic loci showing non-random association with each other\cite{Slatkin2008LinkageFuture}. This mean that  alleles have a combined frequency that deviates from the product of their independent frequencies, i.e. for two alleles A and B, AB and ab are much more common than Ab or aB. Because LD is a population based method, and reflects the historical recombination events in that population, it can be used to construct genetic maps for a given population with high resolution. These maps can be combined with association data from GWASs or similar to show the variants that are genetically close to the trait-associated SNPs. As GWASs cannot distinguish between variants that are inherited together, LD mapping allows for analysis to show the region that is most likely to contain the functional variant\cite{Hutchinson2020Fine-mappingAssociations}. This is the process applied in this study: to find the disease loci that are most likely to be responsible for susceptibility to sPD, and test if these loci significantly affect gene expression.

\subsubsection{Extracting lead SNP information}
The summary statistics for each of the 90 lead SNPs was extracted from the Nalls et al. 2019 paper\cite{Nalls2019IdentificationStudies}. The physical coordinates of the lead SNPs was in the human genome build 37 format (GRCh37). In order to locate the genetic position of the lead SNPs, a European-population fine mapping dataset was used\cite{Maniatis2004PositionalDisequilibrium.} which contained the genetic (in cumulative linkage disequilibrium units (LDU)) and physical coordinates (in kb) of 2110487 variants. This map was in build 35 (NCBI35) of the human genome. To match the datasets together, the UCSC liftover tool\cite{Hinrichs2006The2006.} was used to convert the physical coordinates of the genetic map into build 37 format. 52861 variants were lost in the process, but since this only represents a small fraction of the overall dataset (2.5\%) this was deemed acceptable. The map now included the physical and genetic coordinates of 2057626 variants across the genome. Then, the Nalls et al.\cite{Nalls2019IdentificationStudies} lead SNPs were imported into this dataset. 30 of them were already included and no further action was needed. The other 60 did not have genetic coordinates in this map, so the mean of genetic coordinates of the closest variant upstream and downstream was used as an estimated genetic coordinate for the lead SNP.
\subsubsection{Finding coordinates of LD block}
\label{subsubsec:LDblock}
For most lead SNPs, their "LD block" was defined as all variants in the exact same genetic location, ($\pm0LDU$ from each other). The data file from section INSERTLINKHERE was then filtered to only contain variants with $\pm0LDU$ from each of the lead SNPs. 2 pairs of 2 lead SNPs (rs823118 \& rs11557080, rs62053943 
\& rs117615688) had the same genetic coordinates, and were considered to be part of the same locus, reducing the total loci from 90 to 88. Some lead SNPs did not have any variants $\pm0LDU$ from them, partially as their coordinates had to be estimated, so the definition of an LD block was extended to $\pm0.1LDU$ for these variants. After filtering all variants to only be within the genetic coordinates of the LD block, the physical coordinates of the first and last variant within each of the 88 blocks were used to define the boundaries of the blocks. 

\subsection{Identification of eQTL \textit{cis}-genes}
To address the second aim of this project, two gene expression databases were used for identification and replication of \textit{cis}-genes: the Genototype Tissue Expression (GTEx) project\cite{Lonsdale2013TheProject} and the Brain Expression Almanac (BRAINEAC)\cite{Ramasamy2014GeneticBrain}. Both of these databases are population based, and are not designed around any particular disease. Importantly for this study, they contain expression data and genotype data for each sample - which allows for the calculation of eQTLs (described in section \hyperref[subsubsec:QTL]{2.3.2}).
\subsubsection{GTEx}
\label{subsec:GTEx}
Normalized expression matrices and vcf genotype data were obtained from GTEx version 8. As the latest GTEx v8 data is based on hg38, the coordinates from section \hyperref[subsubsec:LDblock]{3.1.2} were first converted from hg37 using the UCSC liftover tool\cite{Hinrichs2006The2006.}. All variants within the updated coordinates were extracted from the $GTEx_Analysis_2017-06-05_v8_WholeGenomeSeq_838Indiv_Analysis_Freeze.vcf$ (Might be an accession code somewhere? Would be better than the filename!) using vcftools\cite{Danecek2011TheVCFtools} to create a vcf file that contained genotype for all variants within the LD blocks for all 838 samples in GTEx. After subsetting to European samples only (715 of the 838 samples), using the phenotype information file (again, probably an accession somewhere), LD statistics were calculated for the variants in relation to the lead SNP of their block.  This is performed in R using a correlation test. The variants with $R^2\geq0.8$ were retained for further analysis. For the two blocks with two lead SNPs, the higher of the two $R^2$ estimates was taken for the cutoff. This list of variants was screened through ANNOVAR\cite{Wang2010ANNOVAR:Data} to categorize them and search for any protein-altering mutations.
A total of 114 samples were available for SN expression data. Of these 114 samples, 101 were of European descent, and  7 of those 101 had either Alzheimer's or dementia, so were removed from this analysis. The remaining 94 samples were used in the analysis. The genotype file was subsetted to only contain the information for these 94 samples. The expression and genotype files were then converted to a suitable format, and The MatrixEQTL R package\cite{Shabalin2012MatrixOperations.} was then used to to calculate eQTLs between each of the variants in high LD with the lead SNPs and the expression data for genes within 1.5Mb of the disease loci. (Should i state the rationale for using 1.5Mb instead of standard 1Mb?). A nominal significance of $p\leq0.05$ was used, as the prior genetic design and co-localization reduced multiple testing. Four covariates were included in the eQTL calculation: age, sex, ischemic time of sample, and smoking status. Age and sex were included as Nalls et al.\cite{Nalls2019IdentificationStudies} used these as part of their original design. Smoking status was used as it is known to be inversely correlated with PD\cite{Ben-Shlomo2024TheDisease}. Ischemic time was included as a linear regression model fitted using \textit{limma}\cite{Ritchie2015LimmaStudies} showed a significant correlation between ischemic time and the expression values of the sample.

\subsubsection{BRAINEAC}
To replicate the eQTL results from GTEx, another database was used: BRAINEAC\cite{Ramasamy2014GeneticBrain}, the data portal of the UK Brain Expression Consortium. It is a database exclusively focused on the brain, and has samples from 10 different tissues. Genotype and expression data was available for 100 SN samples from Europeans. The expression data was obtained in fastq format, which was was first aligned using STAR\cite{Dobin2013STAR:Aligner} to the transcriptome. After alignment, PCA analysis was performed. One sample was observed as an outlier and was removed from the analysis
RSEM\cite{li211} was used to calculate TPM for the samples
normalised expression scores were generated using ????? to match the format of the GTEx data. NOT FINISHED AS ANALYSIS NOT PERFORMED YET!
\\eQTLs were calculated in the way as the GTEx data, aside from the lack of smoking status as a covariate, as it was not available.
After the lists of significant \textit{cis}-gene variant associations were created, each list was reduced to the most significant result for each gene, as this study is not attempting to identify causal variants. The two lists of unique significant eQTL \textit{cis}-genes were then compared for overlap, and then combined for further analysis. 
\subsection{Validation of \textit{cis}-genes}
The previous section searched for associations between the disease loci and \textit{cis}-genes in healthy population data. In order to then validate the identified eQTL \textit{cis}-genes for relevance to sPD, Differential Gene Expression analysis (DGE) was performed on the list. Two databases were used for DGE: the Gene Expression Omnibus (GEO), and FOUNDIN-PD. Both databases contain expression data, stratified for sPD status. This allows each gene to be tested for a significant difpretference in expression between the case samples and the control samples. 
\subsubsection{GEO}
The GEO \cite{Barrett2012NCBISetsupdate} is a public repository for genomics data. This provides a platform for large-scale analysis of observational data for many diseases. As part of previous work by the group, a meta-analysis of 6 PD case-control expression datasets following the methodology described by Choi et al.\cite{Choi2003CombiningVariation} and using the GeneMeta package\cite{LusaL2024GeneMeta:Experiments.}. The datasets were selected based on (need to get the details), and were all specifically sPD, not fPD. Information about the samples are summarised in table ?(need to implement other tables first).
\begin{table}[h]
    \hskip-1.5cm
    \begin{tabular}{ |c|c|c|c|c|c|c|c|c| }
        \hline
        Accession & Tissue & \# of Controls & \# of Cases & Age & Sex & PMI & RIN \\
        GSE106608 & Subthalamic nucleus & 9 & 7 & Y & Y & N & N \\
        GSE133101 & Amygdala & 26 & 43 & N & N & N & N \\
        GSE136666SN & Substantia nigra & 5 & 5 & N & Y & N & N \\
        GSE205450CAU & Caudate & 40 & 35 & Y & Y & Y & Y \\
        GSE205450PUT & Putamen & 41 & 34 & Y & Y & Y & Y \\
        GSE68719 & Prefrontal cortex & 44 & 29 & Y & Y & Y & Y \\
         \hline
    \end{tabular}
    \caption{Summary information for the 6 studies used for meta-analysis, including: accession ID of the study; tissue from which samples were taken; numbers of cases and controls; availability of covariates age, sex, Post Mortem Index (PMI) and RNA Integrity Number (RIN).}
    \label{tab:my_label}
\end{table}
\\From these 6 studies, meta-p values (nominal and adjusted) and meta-z scores were calculated for expression between cases and controls. The list of significant eQTL \textit{cis}-genes was then filtered through the results, to select only the genes which had nominally significant ($p\leq0.05$) differential expression between sPD cases and controls.
\subsubsection{PPMI}
The Parkinson's Progression Markers Initiative (PPMI)\cite{Marek2011ThePPMI} is a large-scale study with multiple sub-studies. The data used in this study was accessed from FOUNDIN-PD\cite{Bressan2023TheMechanism}, which is a repository for induced Pluripotent Stem Cell (iPSC) data, differentiated from blood samples of PD patients and from healthy controls. The samples used were restricted to just those differentiated into dopaminergic neurons, to model the SN that is the main focus of this study. The final time point, day 95, was used as the analysis date. Additionally, carriers of LLRK2, GBA1 or SNCA mutations were removed. The final number of samples was 37. These 37 scRNA samples were then combined to form a pseudobulk dopaminergic neuron dataset. (need to find info / ask Rahel about more detail on pseudobulk and this procedure in general). Once the proccessing of the scRNA data was complete, DGE was performed on the cases against the controls using \textit{limma}\cite{Ritchie2015LimmaStudies}. The list of genes validated in GEO was then further filtered to just those that showed significant evidence ($p\leq0.05$) of differential expression in the experimental iPSC dataset. This created the final list of genes that showed significant evidence of differential expression by lead-SNP associated disease loci, by observational sPD case-control data, and by experimental sPD case-control data. These were then taken forward for pathway analysis.
\subsection{Pathway Analyses}
After establishing the list of significant disease-eQTL-\textit{cis}-genes, this list can be used for multiple different analyses. The first analysis was to classify all genes within this list as mitochondrial or not based on the Mitocarta3.0 database\cite{Rath2021MitoCarta3.0:Annotations} of known mitochondrial genes. Following the methodology in \cite{Maude2021NewDiabetes.}, these would be used to perform Gene Set Enrichment Analysis\cite{Subramanian2005GeneProfiles}(GSEA). GSEA ranks all the genes in an input, and then tests if there is significant evidence of the query list of genes being clustered at one end (that is, the query set of genes being either up or down regulated compared to the background). GSEA was performed with the genomic background from (which dataset is this being performed in?) against: 43 mitochondrial pathways downloaded from MsigDB\cite{Liberzon2011Molecular3.0}, chosen based on their members having a high level of overlap ($\geq25\%$) with Mitocarta\cite{Rath2021MitoCarta3.0:Annotations}, all significant disease-eQTL \textit{cis}-DEGs, and just the significant cis\textit{DEGs} that are NEMGs. Additionally, a test of \textit{cis}-DEG NEMGs against the background of all NEMGs was performed.
\section{Results}
\subsection{LD block locations}


SUMMARY TABLE 1:
\begin{table}[h]
    \hskip-1.5cm
    \begin{tabular}{ |c|c|c|c|c|c|c|c|c| }
        \hline
        
         \hline
    \end{tabular}
LENGTH KB  NUMBER OF VARIANTs (+high LD)  #CIS genes
Min
mean
median
Max
hist of length


STEP DIAGRAM OF AN EXAMPLE BLOCK


\subsection{eQTL results}
Combined, for GTEx and BRAINEAC:
Summary for number of cisgenes
Signif cisgenes all cisgenes percentage
max
min
mean
median
hist of signif cisgenes


\subsection{DGE results}
Summary for all cisgenes, are the DEGs or not:
Initial, GEO, PPMI, overlap


MEGA venn diagram of ALL cis-genes across BRAINEAC PPMI GEO GTEX

\subsection{Pathway analysis results}
Summary information for the blocks

using fgsea
GSEA of mito 43 + NEMG lists (from hannah paper)
Also on KEGG/similar

Some way of showing go terms/pathway of the genes



\section{Discussion}

strengths and limitations
future work
conclusion
Talking
Non-identification of functional variants
Different study designs in the various databases used. In particular, the observational data from GEO and experimental data from PPMI 

The observed data from GEO is not restricted to just the SN. The SN is the smallest subsection of the samples. 
There is heterogeneity between each of the different experiments used from GEO, although there has been a lot of work by the lab to ensure the most comparable` datasets were used. 
This may impact which genes are differentially expressed.

Some talk about sQTL vs the eQTL



\bibliographystyle{unsrt}
\bibliography{Thesis/references.bib}
\end{document}


\subsubsection{Finding all SNPs in block ###DEPRECIATED???###}
The package LDlinkR's (citation) LDproxy function was used to extract the information for all variants in the database around each of the lead SNPs from the Nalls paper (citation) that were in high LD with the lead SNP (\(R^2\geq0.8\)). This generated a file that was then pruned with the physical coordinates of the LD block from section INSERTNUMBERHERE. The final output of this process was a file with INSERTNUMBERHERE variants, each tagged for which of the 88 LD blocks they came from. This failed for one of the lead SNPs (INSERTRSIDHERE), as it was classified as a functional variant, so proxies could not be found.

Initially, a preliminary analysis on the publicly accessible data in the GTEx database (citation) was completed. This was to ensure that usable results could be generated from the data. This preliminary search was targeted on the \textit{cis}-gene eQTL files provided, and specifically in the SN. This file was subsetted and combined with the file containing all the variants found within the LD blocks to generate a list of eGenes that were within 1mb of the variant and were significantly affected by it.

A total of 2650 (provisional number) genes within 1.5Mb of the disease loci were identified as being significantly affected (p\leq0.05) by at least one variant within the loci that was in high LD with the lead SNP, from the GTEx project data, in the SN. When the same process was applied to the BRAINEAC dataset, a total of ????? genes were identified. ???? genes were significant in both GTEx and BRAINEAC, and a total list of ????? was generated, which contained significant genes in either dataset. The number of significant genes for each block is displayed below:
TABLE????? SUMMARY INFORMATION